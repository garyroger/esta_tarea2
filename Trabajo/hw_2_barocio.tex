\documentclass[conference]{IEEEtran}

\usepackage{graphicx}
\usepackage{subfigure}
\usepackage{amsmath}
\usepackage{amssymb} 
\usepackage{array}

\usepackage{blindtext}
\usepackage{color}

\usepackage[colorlinks=true, linkcolor=blue, citecolor=blue, urlcolor=blue]{hyperref}
\usepackage[all]{hypcap}
\usepackage{float}

\usepackage{etoolbox}
\makeatletter
\patchcmd{\@eqnnum}{\hb@xt@.01\p@}{\hypertarget{equation.\theequation}{}}{}{}
\makeatother

\def\BibTeX{{\rm B\kern-.05em{\sc i\kern-.025em b}\kern-.08em
    T\kern-.1667em\lower.7ex\hbox{E}\kern-.125emX}}

\usepackage{mwe}
\usepackage{fancyhdr}
\fancypagestyle{firststyle}
{
	\fancyhf[C]{\fontsize{8}{10} \selectfont \textit{} }
	\fancyfoot[C]{}
}

%\hyphenation{ }

\begin{document}

% Do not put math or special symbols in the title.
\title{Determinación de parametros del generador síncrono usando pruebas de simulación de rechazo de carga}


\author{
\IEEEauthorblockN{José de Jesús Reyes Ramírez\IEEEauthorrefmark{1},
Yosniel\IEEEauthorrefmark{2},
Luis\IEEEauthorrefmark{3},
Gary\IEEEauthorrefmark{4},
Aylem\IEEEauthorrefmark{5}}
\IEEEauthorblockA{\textit{Universidad de Guadalajara, CUCEI} \\
\textit{Maestría en Ciencias en Ingeniería Eléctrica} \\
Guadalajara, Jalisco, México \\
\IEEEauthorrefmark{1}jose.reyes0963@alumnos.udg.mx,
\IEEEauthorrefmark{2}@alumnos.udg.mx,
\IEEEauthorrefmark{3}@alumnos.udg.mx,
\IEEEauthorrefmark{4}a@alumnos.udg.mx,
\IEEEauthorrefmark{5}@alumnos.udg.mx}
}


\maketitle

\thispagestyle{firststyle}
\renewcommand{\headrulewidth}{0in}
\pagestyle{empty}

\pagestyle{fancy}
\chead{\fontsize{8}{10} \selectfont \textit{} }
\pagenumbering{gobble}

\begin{abstract}
	En el presente se muestra como obtener los parámetros estándar del generador síncrono a partir simulaciones de pruebas de rechazo de carga en Matlab/Simulink.
\end{abstract}

\section{Introducción}
El generador síncrono es \dots

Los parámetros fundamentales del generador síncrono son: $r_s$, $r_{fd}$, $r_{kd}$, $r_{kq1}$, $r_{kq2}$, $x_{Ls}$, $x_{1fd}$, $x_{lkd}$, $x_{lkq1}$, $x_{lkq2}$, $x_{md}$ y $x_{mq}$. Los cuales se calculan usando relaciones matemáticas.

Los parámetros estándar del generador síncrono son las reactancias síncronas, las reactancias síncronas transitorias, las reactancias síncronas subtransitorias, las constantes de tiempo transitorias y subtransitorias en circuito abierto y las constantes de tiempo transitorias y subtransitorias en cortocircuito; es decir, $x_d$, $x_q$, $x^{'}_d$, $x^{'}_q$, $x^{''}_d$, $x^{''}_q$, $T^{'}_{do}$, $T^{'}_{qo}$, $T^{''}_{do}$, $T^{''}_{qo}$, $T^{'}_{d}$, $T^{'}_{q}$, $T^{''}_{d}$, $T^{''}_{q}$.




\section{Conclusiones}
Se concluye que 

\begin{thebibliography}{plain}
\bibitem{b1} M. H. Rashid, Power Electronics: Devices, Circuits \& Applications, Pearson Education, 2013.
\bibitem{b2} Mohan N., Undeland T. M., Robbins W. P. (2002) Power Electronics: Converters, Applications, and Design, Wiley.
\end{thebibliography}

\end{document}